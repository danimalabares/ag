\input{/Users/daniel/github/config/preamble.sty}

\begin{document}

\begin{minipage}{\textwidth}
	\begin{minipage}{1\textwidth}
		Geometry of Algebraic Varieties \hfill Daniel González Casanova Azuela
		
		{\small\hfill\href{https://github.com/danimalabares/ag}{github.com/danimalabares/ag}}
	\end{minipage}
\end{minipage}\vspace{.2cm}\hrule

\vspace{10pt}

{\Huge Exercises in algebraic geometry}

If not explicity stated, exercises are from Hartshorne.

\tableofcontents

\section{Chapter I}

\addcontentsline{toc}{subsection}{Exercise 1.1}
\begin{manualexercise}{1.1}[My first algebraic variety]
	\begin{enumerate}[label*=(\alph*)]\leavevmode
		\item Let $Y$ be the plane curve $y=x^2$ (ie., $Y$ is the zero set of the polynomial $f=y-x^2$). Show that $A(Y)$ is isomorphic to a polynomial ring in one variable over $k$.
		\item Let $Z$ be the plane curve $xy=1$. Show that $A(Z)$ is not isomorphic to a polynomial ring in one variable over $k$.
		\item[*(c)] Let $f$ be any irreducible quadratic polynomial in $k[x,y]$, and let $W$ be the conic defined by $f$. Show that $A(W)$ is isomorphic to $A(Y)$ or $A(Z)$. Which one is it when?
	\end{enumerate}
\end{manualexercise}

\begin{proof}\leavevmode
	\begin{enumerate}[label=(\alph*)]
		\item Consider the map
		\begin{align*}
			k[x,y]&\to k[x]\\
			1&\mapsto1\\
			x&\mapsto x\\
			y&\mapsto x^2
		\end{align*}
		Notice that $y-x^2\in k[x,y]$ is mapped to $0$, so the kernel of this map is $(y-x^2)$. It is also surjective, so we have $A(Y)=k[x,y]/(y-x^2)\cong k[x]$.
		\item In constructing a map like in the former exercise, we may fix $1$ and $x$, and we should map $y$ to $1/x$. However, $1/x$ is not an element of $k[x]$ so we really have an isomorphism $k[x,y]/(xy-1)\cong k[x,\frac{1}{x}]\not\cong k[x]$.
	\end{enumerate}
\end{proof}

\addcontentsline{toc}{subsection}{Exercise 2.14}
\begin{manualexercise}{2.14}[The Segre Embedding]
	Let $\psi:\mathbb{P}^r\times\mathbb{P}^s\to\mathbb{P}^N$ be the map defined by sending the order pair $(a_0,\ldots,a_r)\times(b_0,\ldots,b_s)$ to $(\ldots,a_ib_j,\ldots)$ in lexicographic order, where $N=rs+r+s$. Note that $\psi$ is well-defined and injective. It is called the \textbf{\textit{Segre embedding}}. Show that the image of $\psi$ is a subvariety of $\mathbb{P}^N$. [\textit{Hint}: Let the homogeneous coordinates of $\mathbb{P}^N$ be $\{z_{ij}:i=0,\ldots,r,j=0,\ldots,s\}$ and let $\mathfrak{a}$ be the kernel of the homomorphism $k[\{z_{ij}\}]\to k[x_0,\ldots,x_r,y_0,\ldots,y_s]$ which sends $z_{ij}$ to $x_iy_j$. Then show that $\operatorname{img}\psi=Z(\mathfrak{a})$.
\end{manualexercise}

\begin{proof}[Solution]
	First let's make sure the dimension $N$ is correct. The easy way is found in \href{https://en.wikipedia.org/wiki/Segre_embedding}{wiki}: $N=(r+1)(s+1)-1$ which is the number of possible choices of pairs of things taking one out $r+1$, another out of $s+1$, and then remember there is only one zero index so take one away.
	
	To see that $\psi$ is injective we follow \href{https://math.stackexchange.com/questions/3683364/segre-map-is-an-embedding}{StackExchange}: 
	{\color{azure}Let $z=[z_{00}:z_{01}:\ldots:z_{ij}:\ldots:z_{rs}]$ be an element of the image of $\psi$ and let $(a,b)\in\mathbb{P}^r\times\mathbb{P}^s$ be such that $\psi(a,b)=z$. WLOG we can assume $a_0=b_0=z_{00}=1$. Then $b_j=z_{0j}$ for all $0\leq j\leq s$ and $a_i=z_{i0}$ so $a,b$ are uniquely determined and this map is bijective onto the image.
	
	Actually, what we have done is constructed an inverse morphism of the Segre map. According to StackExchange, this makes it into an embedding.}
	
	To show that $\operatorname{img}\psi$ is a subvariety of $\mathbb{P}^N$ we need to find a set of homogeneous polynomials in $k[z_{ij}]$/
	
	Following the hint, as before let $z\in\operatorname{img}\psi$ and $f$ any polynomial in the kernel of \[k[\{z_{ij}\}]\to k[x_0,\ldots,x_r,y_0,\ldots,y_s]\]. We must show that $f(z)=0$. Well it doesn't make much sense because if $f=\sum a_{ij}z_{ij}$ is in the kernel of that map, then its image $\sum a_{ij}x_iy_j$ is the zero polynomial, so obviously $f(z)=\sum a_{ij}z_{ij}=\sum a_{ij}x_iy_j=0$. So this is confusing.
	
	So what are the equations of $\operatorname{img}\psi$? A polynomial $f(z_{00},\ldots,z_{rs})$ will vanish on $\operatorname{img}\psi$ if somehow it vanishes 
\end{proof}

\addcontentsline{toc}{subsection}{Exercise 2.15}
\begin{manualexercise}{2.15}[The Quadric Surface in $\mathbb{P}^3$]
	Consider the surface $Q$ (a \textbf{\textit{surface}} is a variety of dimension 2) in $\mathbb{P}^3$ defined by the equation $xy-wz=0$.
	\begin{enumerate}
		\item Show that $Q$ is equal to the Segre embedding of $\mathbb{P}^1\times\mathbb{P}^1$ in $\mathbb{P}^3$, for suitable choice of coordinates.
		\item Show that $Q$ contains two families of lines (a \textbf{\textit{line}} is a linear variety of dimension 1), $\{L_t\},\{M_t\}$ each parametrized by $t\in\mathbb{P}^1$, with the properties that if $L_t\neq L_u$ then $L_t\cap L_u=\varnothing$ and if $M_t\neq M_u$, $M_t\cap M_u=\varnothing$, and for all $t,u$, $L_t\cap M_u$ is a point.
		\item Show that $Q$ contains other curves besides these lines, and deduce that the Zariski topology on $Q$ is not homeomorphic via $\psi$ to the product topology on $\mathbb{P}^1\times \mathbb{P}^1$ where each $\mathbb{P}^1$ has its Zariski topology.
	\end{enumerate}
\end{manualexercise}

\begin{proof}[Solution]\leavevmode
	\begin{enumerate}
		\item It turns out that the image of the Segre embedding $\psi:\mathbb{P}^1\times\mathbb{P}^1\to\mathbb{P}^3$ equals is the algebraic variety given by the zeroes of the polynomial $f=z_{00}z_{11}-z_{10}z_{01}\in k[z_{00},z_{01},z_{10},z_{11}]$. One contention is easy: if $(x,y)=([x_0,x_1],[y_0,y_1])\in\operatorname{img}\psi$, then clearly $f(\psi(x,y))=x_0y_0x_1y_1-x_0y_1x_1y_0$ is zero because these are numbers in the field $k$.
		
		Now for the other contention pick $z=[z_{00},z_{01},z_{10},z_{11}]\in V(f)$ and let's find an element $(x,y)\in\mathbb{P}^1\times\mathbb{P}^1$ such that $\psi(x,y)=z$. $z\in V(f)$ means that $z_{00}z_{11}=z_{10}z_{01}$. If $z_{00}\neq0$, then we can define $([z_{00},z_{11}],[z_{01},z_{10}])$ {\color{magenta}what?}
		%so there is always one of $z_{00}$ or $z_{11}$ and one of $z_{10}$ or $z_{01}$ that are not zero.
		
		Maybe for the other contention try to define the inverse map $\operatorname{img}\psi\to\mathbb{P}^1\times\mathbb{P}^1$ by $z=[z_{00},z_{01},z_{10},z_{11}]\mapsto([z_{00},z_{01}],[z_{00},z_{10}])$ when $z_{00}\neq0$ and $([z_{11},z_{01}],[z_{11},z_{10}])$ when $z_{11}\neq0$. Is this defining a global map?
		
		\item The lines correspond to fixing one entry and running over the other one in the Segre embedding $(x,y)\to z$. 
	\end{enumerate}
\end{proof}

\addcontentsline{toc}{subsection}{Exercise 3.16}
\begin{manualexercise}{3.16}[Products of Quasi-Projective Varieties]
	Use the Segre embedding (Ex. 2.14) to identify $\mathbb{P}^n\times\mathbb{P}^m$ with its image and hence give it a structure of projective variety. Now for any two quasi-projective varieties $X\subseteq\mathbb{P}^n$ and $Y\subseteq\mathbb{P}^m$ consider $X\times Y\subseteq\mathbb{P}^n\times\mathbb{P}^m$.
	\begin{enumerate}[label*=(\alph*)]
		\item Show that $X\times Y$ is a quasi-projective variety.
		\item If $X,Y$ are both projective, show that $X\times Y$ is projective.
		\item Show that $X\times Y$ is a product in the category of varieties.
	\end{enumerate}
\end{manualexercise}

\begin{proof}[Solution]
	content...
\end{proof}

\section{[Ot] Chapter I: Varieties}

A friend from Cabo Frio just recommended me to have a look at \href{https://www.uio.no/studier/emner/matnat/math/MAT4215/data/masteragbook.pdf}{Ottem\&Ellinsburg, Introduction to Schemes}. Here are some exercises I liked from Chapter 1: Varieties.

\addcontentsline{toc}{subsection}{Exercise 1.5.12}
\begin{manualexercise}{1.5.12}[The diagonal]
	Let $X$ be an affine variety and consider the map
	\begin{align*}
		\Delta: X &\longrightarrow X\times X \\
		x &\longmapsto (x,x)
	\end{align*}
	\begin{enumerate}[label=\alph*.]
		\item Show that $\Delta$ is a polynomial map.
		\item Let $X=\mathbb{A}^{n}(k)$…
		\item …gives an isomorphism $X\to \Delta(X)$. Hint…
	\end{enumerate}
\end{manualexercise}

\addcontentsline{toc}{subsection}{Exercise 1.5.15}
\begin{manualexercise}{1.5.15}
{\color{magenta}Some Lie groups that are algebraic sets}
\end{manualexercise}

\addcontentsline{toc}{subsection}{Exercise 1.5.28}
\begin{manualexercise}{1.5.28}
	Show that the image of the map
	\begin{align*}
		\phi: \mathbb{A}^{1}(k) &\longrightarrow \mathbb{A}^{3}(k) \\
		t &\longmapsto (t^{2},t^{3},t^{6})
	\end{align*}
	is given by $V(x^{3}-y^{2},z-x^{3})$. Show that $\phi$ is bijective. Is $\phi$ an isomorphism of affine varieties.
\end{manualexercise}

\addcontentsline{toc}{subsection}{Exercise 1.5.29}
\begin{manualexercise}{1.5.29}
	Show that the image of the map
	\begin{align*}
		\phi: \mathbb{A}^{1}(k) &\longrightarrow \mathbb{A}^{3}(k) \\
		t &\longmapsto (t^{3},t^{4},t^{5})
	\end{align*}
	is given by $V(x^{4}-y^{3},z^{3}-x^{5},y^{5}-z^{4})$. Show that $\phi$ is bijective. Is $\phi$ an isomorphism of affine varieties.
\end{manualexercise}

\addcontentsline{toc}{subsection}{Exercise 1.5.31}
\begin{manualexercise}{1.5.31}
	Show that the image of the map
	\begin{align*}
		\phi: \mathbb{P}^{1}(k) &\longrightarrow \mathbb{P}^{2}(k) \\
		(x_0:x_1( &\longmapsto (x_0^{2},x_0x_1,x_1^{2})
	\end{align*}
	is given by $V(y_1^{2}-y_0y_2)$. Show that $\phi$ is an isomorphism of projective varieties. Deduce that any projective conic is isomorphic to $\mathbb{P}^{1}(k)$.
\end{manualexercise}

\section{Chapter IV}

\addcontentsline{toc}{subsection}{Exercise 1.2}
\begin{manualexercise}{1.2}[I like this one]
	Again let $X$ be a curve, and let $P_1,\ldots,P_r$ be points. Then there is a rational  function $f\in K(X)$ having poles (of some order) at each of the $P_i$ and regular  elsewhere.  
\end{manualexercise}

\addcontentsline{toc}{subsection}{Exercise 1.7}
\begin{manualexercise}{1.7}[no one]
	A curve $X$ is called \textit{\textbf{hyperelliptic}}…
\end{manualexercise}

\addcontentsline{toc}{subsection}{Exercise 1.8}
\begin{manualexercise}{1.8}[Alex]
	Very useful to know, I think this is done in that book by Bosch of modules,
\end{manualexercise}

\addcontentsline{toc}{subsection}{Exercise 1.9}
\begin{manualexercise}{1.9}[Victor]
	Riemann-Roch for singular curves.
\end{manualexercise}

\addcontentsline{toc}{subsection}{Exercise 2.3(h)}
\begin{manualexercise}{2.3(h)}
	28 bitangents. Remind Sergey.
\end{manualexercise}

\addcontentsline{toc}{subsection}{Exercise 2.5}
\begin{manualexercise}{2.5}
	Prove the theorem of Hurwitz that a curve $X$ of genus $g\geq 2$ over a field of characteristic 0 has at most $84(g-1)$.
\end{manualexercise}

\addcontentsline{toc}{subsection}{Exercise 3.1}
\begin{manualexercise}{3.1}
	If X is a curve of genus 2, show that a divisor D is very ample $\iff \operatorname{deg} D \geq 5$.  This strengthens (3,3.4).
\end{manualexercise}

\addcontentsline{toc}{subsection}{Exercise 3.12}
\begin{manualexercise}{3.12}
	For each value of $d = 2,3,4,5$ and $r$ satisfying $0\leq r\leq \frac{1}{2}(d-1)(d-2)$, show  that there exists an irreducible plane curve of degree d with r nodes and no other  singularities.
\end{manualexercise}

\addcontentsline{toc}{subsection}{Exercise 4.10}
\begin{manualexercise}{4.10}
	If $X$ is an elliptic curve (Sergey: for abelian varieties is also true), show that there is an exact sequence… Picard groups.
\end{manualexercise}

\addcontentsline{toc}{subsection}{Exercise 5.3}
\begin{manualexercise}{5.3}
	Moduli of Curves of Genus 4. The hyperelliptic curves of genus 4 form an irreducible family of dimension 7. The nonhyperelliptic ones form an irreducible  family of dimension 9. The subset of those having only one $g_3^{1}$ is an irreducible  family of dimension 8. [Hint: Use (5.2.2) to count how many complete intersections $Q\cap F_3$ there are.]  
\end{manualexercise}

\addcontentsline{toc}{subsection}{Exercise 6.2}
\begin{manualexercise}{6.2}
	A rational curve of degree 5 in $\mathbb{P}^{3}$ is always contained in a cubic surface, but there  are such curves which are not contained in any quadric surface.  
\end{manualexercise}

\section{Chapter V}

\addcontentsline{toc}{subsection}{Exercise 1.8}
\begin{manualexercise}{1.8}
	Divisor cohomology, neron severi
\end{manualexercise}

\addcontentsline{toc}{subsection}{Exercise 2.8}
\begin{manualexercise}{2.8}
	Locally free sheaves.
\end{manualexercise}

\addcontentsline{toc}{subsection}{Exercise 3.5}
\begin{manualexercise}{3.5}
	5 points in the field, hyperelliptic curve, point at infinity is singular.
\end{manualexercise}

\addcontentsline{toc}{subsection}{Exercise 4.6}
\begin{manualexercise}{4.5}
	
\end{manualexercise}

\addcontentsline{toc}{subsection}{Exercise 4.16}
\begin{manualexercise}{4.16}
	27 lines on Fermat cubic
\end{manualexercise}

\addcontentsline{toc}{subsection}{Exercise 5.1}
\begin{manualexercise}{5.1}
	
\end{manualexercise}

\addcontentsline{toc}{subsection}{Exercise 5.4}
\begin{manualexercise}{5.4}
	
\end{manualexercise}

\addcontentsline{toc}{subsection}{Exercise 5.5}
\begin{manualexercise}{5.5}
	
\end{manualexercise}

\addcontentsline{toc}{subsection}{Exercise 6.2}
\begin{manualexercise}{6.2}[Arthur]
	Beautiful exercise.
\end{manualexercise}

\end{document}
